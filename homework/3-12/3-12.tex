\documentclass[a4paper,11pt,twocolumn]{article}
\usepackage{fancyhdr}
\usepackage{enumerate}
\usepackage{pifont}
\usepackage{fontspec, xunicode, xltxtra}
\usepackage{mathptmx}
\usepackage{amsmath}
\usepackage{amsfonts}
\usepackage{amssymb}
\usepackage{graphicx}
\usepackage[top=2cm, bottom=2cm, left=2cm, right=2cm]{geometry}

\usepackage{listings}
\makeatletter
\lst@CCPutMacro\lst@ProcessOther {"2D}{\lst@ttfamily{-{}}{-{}}}
\@empty\z@\@empty
\makeatother
\usepackage{color}
\newfontfamily\mono{Consolas}
\newfontfamily\timesnew{Times New Roman}
\lstset {
  basicstyle = \small\mono,
  columns = fixed,
  extendedchars=false,
  %language = C++,
  tabsize = 4,
  frame = single,
  breaklines = true,
  breakindent = 1.1em,
  showstringspaces=false,
  stringstyle=\mono,
  numberstyle=\footnotesize\mono,
  basewidth={0.5em, 0.4em},
}

\setlength{\columnsep}{7mm}

\newcommand{\homeworkno}{3.12}
\pagestyle{fancy}
\lhead{Problem Solving: Homework \homeworkno}
\chead{}
\rhead{Chen Shaoyuan (161240004)}
\lfoot{}
\cfoot{\thepage}
\rfoot{}

\newcommand{\lcm}{\mathop{\mathrm{lcm}}}
\allowdisplaybreaks[4]
%\renewcommand{\labelenumi}{\textbf{\emph{\alph{enumi}}.}}
\setmainfont{Times New Roman}
\begin{document}
  \title{Problem Solving: Homework \homeworkno}
  \author{Name: Chen Shaoyuan \and Student ID: 161240004}
  \maketitle

  \section{[TJ] Exercise 2-13}
  The First Principle of Mathematical Induction is a special case of the second one, so we only have to prove that the first one implies the second one. \par
  Let $S'(n)$ denote the statement that for every integer $k (n_0 \leq k \leq n)$, $S(k)$ holds. The basis of the second one says that $S(n_0)$ holds, hence $S'(n_0)$ holds. Assume that $S'(n)$ holds, i.e. $S(n_0), S(n_0 + 1), \cdots, S(n)$ hold. By the induction of the second one, $S(n+1)$ holds, so $S'(n+1)$ holds. By the First Principle of Mathematical Induction, $S'(n)$ holds for all $n \geq n_0$, i.e. the Second Principle of Mathematical Induction is true.

  \section{[TJ] Exercise 2-14}
  Assume, to the contrary that 1 is not the smallest natural number. Let $S \neq \varnothing$ denote the set of natural numbers that are less than 1. By the Principle of Well-Ordering, $S$ must contain a smallest number, say $x$. Since $x \neq 1$, by the definition of natural number, it must have a predecessor $x-1$, such that $x-1 < x < 1$, so $x - 1 \in S$. This means that $x$ is not the smallest integer of $S$, which leads to contradiction. Therefore, 1 is the smallest natural number. \par
  Assume, to the contrary that $S \neq \mathbb{N}$, then $\mathbb{N} \setminus S \neq \varnothing$. By the Principle of Well-Ordering, $S$ has a smallest number, say $x$. If $x = 1$, it contradicts the basis of the mathematical induction. If $x \neq 1$, by the contrapositive of the induction, $x-1 \notin \mathbb{N} \setminus S$, which means $x$ is not the smallest number of $\mathbb{N} \setminus S$. The contradiction must occur whatever the value of $x$ is. This means $\mathbb{N} \setminus S \neq \varnothing$, also we have $S \subset \mathbb{N}$, so $S = \mathbb{N}$. Hence the Principle of Mathematical Induction is true.

  \section{[TJ] Exercise 2-15}
  \begin{enumerate}[(a)]
    \item
        \begin{align*}
          39 &= 14 \cdot 2 + 11 \\
          14 &= 11 \cdot 1 + 3 \\
          11 &= 3 \cdot 3 + 2 \\
          3 &= 2 \cdot 1 + 1 \\
          2 &= 1 \cdot 2 + 0
        \end{align*}
        \begin{align*}
          1 &= 3 + (-1) \cdot 2 \\
            &= 3 + (-1) \cdot (11 - 3 \cdot 3) \\
            &= 4 \cdot (14 - 1 \cdot 11) + (-1) \cdot 11 \\
            &= 4 \cdot 14 + (-5) \cdot (39 + (-2) \cdot 14) \\
            &= 14 \cdot 14 + (-5) \cdot 39
        \end{align*}
        $$ \gcd(14, 39) = 1 = 14 \cdot 14 + (-5) \cdot 39 $$
    \item
        \begin{align*}
          562 &= 471 \cdot 1 + 91 \\
          471 &= 91 \cdot 5 + 16 \\
          91 &= 16 \cdot 5 + 11 \\
          16 &= 11 \cdot 1 + 5 \\
          11 &= 5 \cdot 2 + 1 \\
          5 &= 1 \cdot 5 + 0
        \end{align*}
        \begin{align*}
          1 &= 11 + (-2) \cdot 5 \\
            &= 11 + (-2) \cdot (16 + (-1) \cdot 11) \\
            &= 3 \cdot (91 + (-5) \cdot 16) + (-2) \cdot 16 \\
            &= 3 \cdot 91 + (-17) \cdot (471 + (-5) \cdot 91) \\
            &= 88 \cdot (562 + (-1) \cdot 471) + (-17) \cdot 471\\
            &= 88 \cdot 562 + (-105) \cdot 471
        \end{align*}
        $$ \gcd(471, 562) = 1 = (-105) \cdot 471 + 88 \cdot 562 $$
    \item
        \begin{align*}
          234 &= 165 \cdot 1 + 69 \\
          165 &= 69 \cdot 2 + 27 \\
          69 &= 27 \cdot 2 + 15 \\
          27 &= 15 \cdot 1 + 12 \\
          15 &= 12 \cdot 1 + 3 \\
          12 &= 3 \cdot 4 + 0
        \end{align*}
        \begin{align*}
          3 &= 15 + (-1) \cdot 12 \\
            &= 15 + (-1) \cdot (27 + (-1) \cdot 15) \\
            &= 2 \cdot (69 + (-2) \cdot 27) + (-1) \cdot 27 \\
            &= 2 \cdot 69 + (-5) \cdot (165 + (-2) \cdot 69) \\
            &= 12 \cdot (234 + (-1) \cdot 165) + (-5) \cdot 165 \\
            &= 12 \cdot 234 + (-17) \cdot 165
        \end{align*}
        $$ \gcd(234, 165) = 3 = 12 \cdot 234 + (-17) \cdot 165 $$
    \item
        \begin{align*}
          23771 &= 19945 \cdot 1 + 3826 \\
          19945 &= 3826 \cdot 5 + 815 \\
          3826 &= 815 \cdot 4 + 566 \\
          815 &= 566 \cdot 1 + 249 \\
          566 &= 249 \cdot 2 + 68 \\
          249 &= 68 \cdot 3 + 45 \\
          68 &= 45 \cdot 1 + 23 \\
          45 &= 23 \cdot 1 + 22 \\
          23 &= 22 \cdot 1 + 1 \\
          22 &= 1 \cdot 22 + 1
        \end{align*}
        \begin{align*}
          1 &= 23 + (-1) \cdot 22 \\
            &= 23 + (-1) \cdot (45 + (-1) \cdot 23) \\
            &= 2 \cdot (68 + (-1) \cdot 45) + (-1) \cdot 45 \\
            &= 2 \cdot 68 + (-3) \cdot (249 + (-3) \cdot 68) \\
            &= 11 \cdot (566 + (-2) \cdot 249) + (-3) \cdot 249 \\
            &= 11 \cdot 566 + (-25) \cdot (815 + (-1) \cdot 566) \\
            &= 36 \cdot (3826 + (-4) \cdot 815) + (-25) \cdot 815 \\
            &= 36 \cdot 3826 + (-169) \cdot (19945 + (-5) \cdot 3826) \\
            &= 881 \cdot (23771 + (-1) \cdot 19945) + (-169) \cdot 19945 \\
            &= 881 \cdot 23771 + (-1050) \cdot 19945
        \end{align*}
        $$ \gcd(23771, 19945) = 1 = 881 \cdot 23771 + (-1050) \cdot 19945 $$
    \item
        \begin{align*}
          9923 &= 1739 \cdot 5 + 1228 \\
          1739 &= 1228 \cdot 1 + 511 \\
          1228 &= 511 \cdot 2 + 206 \\
          511 &= 206 \cdot 2 + 99 \\
          206 &= 99 \cdot 2 + 8 \\
          99 &= 8 \cdot 12 + 3 \\
          12 &= 3 \cdot 4 + 0 \\
        \end{align*}
        \begin{align*}
          3 &= 99 + (-12) \cdot 8 \\
            &= 99 + (-12) \cdot (206 + (-2) \cdot 99) \\
            &= 25 \cdot (511 + (-2) \cdot 206) + (-12) \cdot 206 \\
            &= 25 \cdot 511 + (-62) \cdot (1228 + (-2) \cdot 511) \\
            &= 149 \cdot (1739 + (-1) \cdot 1228) + (-62) \cdot 1228 \\
            &= 149 \cdot 1739 + (-211) \cdot (9923 + (-5) \cdot 1739) \\
            &= 1204 \cdot 1739 + (-211) \cdot 9923
        \end{align*}
        $$ \gcd(1739, 9923) = 3 = 1204 \cdot 1739 + (-211) \cdot 9923 $$
    \item
        \begin{align*}
          -4357 &= 3754 \cdot (-2) + 3151 \\
          3754 &= 3151 \cdot 1 + 603 \\
          3151 &= 603 \cdot 5 + 136 \\
          603 &= 136 \cdot 4 + 59 \\
          136 &= 59 \cdot 2 + 18 \\
          59 &= 18 \cdot 3 + 5 \\
          18 &= 5 \cdot 3 + 3 \\
          5 &= 3 \cdot 1 + 2 \\
          3 &= 2 \cdot 1 + 1 \\
          2 &= 1 \cdot 2 + 0
        \end{align*}
        \begin{align*}
          1 &= 3 + (-1) \cdot 2 \\
            &= 3 + (-1) \cdot (5 + (-1) \cdot 3) \\
            &= 2 \cdot (18 + (-3) \cdot 5) + (-1) \cdot 5 \\
            &= 2 \cdot 18 + (-7) \cdot (59 + (-3) \cdot 18) \\
            &= 23 \cdot (136 + (-2) \cdot 59) + (-7) \cdot 59 \\
            &= 23 \cdot 136 + (-53) \cdot (603 + (-4) \cdot 136) \\
            &= 235 \cdot (3151 + (-5) \cdot 603) + (-53) \cdot 603 \\
            &= 235 \cdot 3151 + (-1228) \cdot (3754 + (-1) \cdot 3151) \\
            &= 1463 \cdot (-4357 + (-2) \cdot 3754) + (-1228) \cdot 3754 \\
            &= 1463 \cdot (-4357) + 1698 \cdot 3754
        \end{align*}
        $$ \gcd(-4357, 3754) = 1 = 1463 \cdot (-4357) + 1698 \cdot 3754 $$
  \end{enumerate}

  \section{[TJ] Exercise 2-16}
  Suppose that $a$ and $b$ are not relatively prime. Let $g = \gcd(a, b) > 1$, then $a = pg$, $b = qg$, where $p$ and $q$ are integers. Hence $pgr + qgs = g(pr + qs) = 1$. The lhs of the equation is a multiple of $g$, while the rhs is not, which leads to contradiction. Therefore $a$ and $b$ are relatively prime.

  \section{[TJ] Exercise 2-19}
  Let $xy = q^2$. By the Fundamental Theorem of Arithmetic, $x, y$ can be written as
  \begin{align*}
    x &= \prod_{i=1}^{k} {p_i}^{m_i},  &  y &= \prod_{i=1}^{k} {p_i}^{n_i}, & q &= \prod_{i=1}^{k} {p_i}^{s_i}
  \end{align*}
  where $p_i$ is the $i$th prime, $m_i, n_i, s_i$ are nonnegative integers. \par
  $xy = q^2$ implies $2s_i = m_i + n_i$. Since $x$ and $y$ are relatively prime, we have $m_i n_i = 0$. Hence $m_i$, $n_i$ are even, which means $x$ and $y$ are perfect squares.

  \section{[TJ] Exercise 2-22}
  For every integer $m$, by the division algorithm, there exists unique integers $q$ and $t (0 \leq t < n)$, such that $m = nq + t$. So every integer is congruent mod $n$ to precisely one of the integers $0, 1, \cdots, 1$. This means that if $r$ is an integer, there exists unique $s \in \mathbb{Z}$ such that $0 \leq s < n$ and $[r] = [s]$. The union of $[0], [1], \cdots, [n-1]$ is $\mathbb{Z}$, and any two of them are disjoint. So the integers are partitioned by congruence mod $n$.

  \section{[TJ] Exercise 2-28}
  Note that $2^p - 1 = 1 + 2 + 4 + \cdots + 2^{p-1}$. If $p$ is not prime, i.e. $p = qr$, where $q, r \geq 2$, then
  \begin{align*}
    2^p - 1 =& 1 + 2 + \cdots + 2^{q-1} + \\
             & 2^q + 2^{q+1} + \cdots + 2^{2q-1} + \\
             & \cdots \\
             & 2^{q(r-1)} + 2^{q(r-1)+1} + \cdots + 2^{qr-1} \\
            =& (1 + 2^q + \cdots + 2^{q(r-1)}) (1 + 2 + \cdots + 2^{q-1})
  \end{align*}
  is not a prime, which leads to contradiction.

  \section{[TJ] Exercise 2-29}
  Assume, to the contrary that there are finitely many primes of the form $6n+5$, and let $p_1, p_2, \cdots, p_k$ denote them. Every odd prime is either of the form $6n+1$ or $6n+5$. Consider the number $P = p_1p_2 \cdots p_k + 6$, which is congruent to 5 modulo 6. $P$ is not a multiple of any $p_i$ because $P$ is congruent to $6$ modulo $p_i$, while $6$ can't be a multiple of $p_i$. This means, $P$ is the product of several primes of the form $6n+1$, but this means that $P$ is congruent to $1$ modulo 6, which leads to contradiction. Therefore, there are an infinite number of primes of the form $6n+5$.

  \section{[TJ] Exercise 2-30}
  Assume, to the contrary that there are finitely many primes of the form $4n-1$, and let $p_1, p_2, \cdots, p_k$ denote them. Every odd prime is either of the form $4n+1$ or $4n-1$. Consider the number $P = 4p_1p_2p_3 \cdots p_k - 1$, which is congruent to -1 modulo 4. If $P$ a multiple of any $p_i$, then $P + 1$ is also a multiple of $p_i$, this means 1 is a multiple of $p_i$ which is impossible. Hence, $P$ is the product of several primes of the form $4n+1$, but this means that $P$ is congruent to 1 modulo 4, which leads to contradiction. Therefore, there are an infinite number of primes of the form $4n-1$.

  \section{[TJ] Exercise 2-30}
  Suppose to the contrary that there exists integers $p, q$ such that $p^2 = 2q^2$. By the Fundamental Theorem of arithmetic, $p^2$ and $q^2$ can be written as
  \begin{align*}
    p^2 &= \prod_{i=1}^{k} {P_i}^{2m_i},  &  q^2 &= \prod_{i=1}^{k} {P_i}^{2n_i}
  \end{align*}
  where $P_i$ is the $i$th prime, $m_i, n_i$ are nonnegative integers. Since $p^2 = 2q^2$, we have $2m_1 = 2n_1 + 1$, which leads to contradiction. Hence there do not exist such integers $p, q$. \par
  By rewriting $p^2 = 2q^2$, we know that $\sqrt{2} = p/q$. However, we have proved that there do not exists such integers $p, q$, so $\sqrt{2}$ is not a rational number.

  \section{[TJ] Programming Exercise 2-1}
  The prime factors of 120 are 2, 3, 5. By crossing out the multiples of 2, 3, 5, we obtain the integers that are relatively prime to 120. The ones that are less than 120 are listed below.
  \begin{tabbing}
    1111 \= 1111 \= 1111 \= 1111 \= 1111 \= 1111 \= 1111 \= 1111 \= 1111 \= 1111 \kill
    % \> for next tab, \\ for new line...
    2 \> 3 \> 5 \> 7 \> 11 \> 13 \> 17 \> 19 \> 23 \> 29 \\
    31 \> 37 \> 43 \> 47 \> 49 \> 53 \> 59 \> 61 \> 67 \> 71 \\
    73 \> 77 \> 79 \> 83 \> 89 \> 91 \> 97 \> 101 \> 103 \> 107 \\
    109 \> 113 \> 119
  \end{tabbing}
  Any one of these plus a multiple of 120 is also a number relatively prime to 120. \par
  The computer program that computes all primes less than $N$ is shown below.
  \lstinputlisting[language=c]{src/sieve.c}

  \section{[TJ] Programming Exercise 2-3}
  \lstinputlisting[language=c]{src/exgcd.c}

  \section{[CS] Exercise 2.2-2}
  Yes. It is 133.

  \section{[CS] Exercise 2.2-4}
  There is one element, $a = 24$, such that $a \cdot_{31} 22 = 1$. \par
  There is no element such that $a \cdot_{10} 2 = 1$.

  \section{[CS] Exercise 2.2-6}
  All possible common divisors are $1$ and $-1$.

  \section{[CS] Exercise 2.2-8}
  $\gcd(q, k) = \gcd(q, k-jq) = \gcd(r, q)$.

  \section{[CS] Exercise 2.2-15}
  $\gcd(j, k)$ is a factor of $\gcd(r, k)$.

  \section{[CS] Exercise 2.2-16}
  We can take $q' = -q-1$, such that $m = q'n + r$. \par
  If $m \geq 0$, it is identical to Theorem 2.1. If $m < 0$, by Theorem 2.12, there exists unique integers $q, r (0 \leq r < n)$, such that $-m = nq + r$. If $r = 0$, we take $q' = q$ and $r' = 0$, such that $m = nq' + r'$, and such choice is unique. If $r \neq 0$, we take $q' =  - q - 1$, $r' = n - r$, such that $m = nq' + r'$. Here, the choice is also unique, because $q = -q' - 1$, $r = n - r'$, which means two different pairs of $q', r'$ lead to two different pairs of $q, r$, which contradicts Theorem 2.12. Hence we've extended Theorem 2.12 to Theorem 2.1.

  \section{[CS] Exercise 2.2-17}
  Since $F_i = F_{i-1} + F_{i-2}$ for $i \geq 3$, the algorithm performs Euclid's division on $(F_{i+1}, F_{i})$, $(F_{i}, F_{i-1})$, $\cdots$, $(F_2, F_1)$, and the GCD is $1$. And, $1 = 1 \cdot 2 + (-1) \cdot 1 = 2 \cdot 2 - 1 \cdot 3 = 2 \cdot 5 - 3 \cdot 3 = \cdots = (-1) ^ i F_{i-1} F_{i+1} - (-1)^i F_i F_i$. So $\gcd(F_{i+1}, F_i) = 1 = (-1)^i F_{i-1} F_{i+1} + (-1)^{i+1} F_i F_i$.

  \section{[CS] Exercise 2.2-19}
  $ \lcm(a, b) = ab/\gcd(a, b)  $.
\end{document}
