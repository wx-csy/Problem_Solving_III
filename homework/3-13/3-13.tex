\documentclass[a4paper,11pt,twocolumn]{article}
\usepackage{fancyhdr}
\usepackage{enumerate}
\usepackage{times}
\usepackage{mathptmx}
\usepackage{amsmath}
\usepackage{amsfonts}
\usepackage{amssymb}
\usepackage{tikz}
\usepackage{clrscode3e}
\usepackage[top=2cm, bottom=2cm, left=2cm, right=2cm]{geometry}

\setlength{\columnsep}{7mm}

\newcommand{\homeworkno}{3.13}
\pagestyle{fancy}
\lhead{Problem Solving: Homework \homeworkno}
\chead{}
\rhead{Chen Shaoyuan (161240004)}
\lfoot{}
\cfoot{\thepage}
\rfoot{}

\allowdisplaybreaks[4]
\renewcommand{\labelenumi}{\textbf{\emph{\alph{enumi}}.}}
\begin{document}
  \title{Problem Solving: Homework \homeworkno}
  \author{Name: Chen Shaoyuan \and Student ID: 161240004}
  \maketitle

  \section{[TC] Problem 31.1-12}
  \begin{codebox}
  \Procname{$\proc{Divide}(a, b, \beta)$}
  \li $b \gets b \ll \beta$
  \li Let $r$ be a $\beta$-bit integer
  \li Let $\langle r_{\beta-1}, \cdots, r_0 \rangle$ be the binary representation of $r$
  \li $i \gets \beta$
  \li \While $i > 0$
  \li \Do $i \gets i - 1$
  \li     $b \gets b \gg 1$
  \li     \If $a \geq b$
  \li     \Do $r_i \gets 1$
  \li         $a \gets a - b$
  \li     \Else $r_i \gets 0$
          \End
      \End
  \li \Return $r$, $a$
  \end{codebox}
  where $\ll$, $\gg$ means logical left and right shift, respectively. The returned $r$ is the quotient, $a$ is the remainder.

  Each elementary operation (assignment, subtraction, comparison, logical shift) on $\beta$-bit integer(s) takes $\Theta(\beta)$ time, and there are $\Theta(\beta)$ operations in total, so this algorithm runs in $\Theta(\beta^2)$ time.

  \section{[TC] Problem 31.1-13}
  \begin{codebox}
  \Procname{$\proc{Convert}(a)$}
  \li Make the length of $a$ a power of 2 by zero padding
  \li $l \gets \attrib{a}{length}$
  \li $p[0] = 2$
  \li $i \gets 2$
  \li \While $2^i < l$
  \li \Do $p[\log_2 i] = p[\log_2 i - 1] \times p[\log_2 i - 1] $
  \li     $i \gets i + 1$
      \End
  \li \Return $\proc{Convert-Rec}(a)$
  \end{codebox}
  
  \begin{codebox}
  \Procname{$\proc{Convert-Rec}(a)$}
  \li $l \gets \attrib{a}{length}$
  \li \If $l \isequal 1$
  \li \Do \Return $a$
      \End
  \li let $a_1, a_2$ be the higher and lower $l/2$ bits
  \li $b_1 \gets \proc{Convert-Rec}(a_1)$
  \li $b_2 \gets \proc{Convert-Rec}(a_2)$
  \li \Return $b_1 \times P[\log_2 l - 1] + b_2$ 
  \end{codebox}
  
  The precalculation of the decimal representations of $2^{2^k}$ takes $\Theta(M(\beta) \log \beta)$ time. Since $\Theta(n^2)$ brute-force multiplication algorithm is known, we may assume that $M(\beta) = O(\beta^2) = \Omega(\beta)$. The running time of the recursion satisfies the following recurrence
  $$ f(n) = 2f(n/2) + M(n) $$
  since $kM(n/k) = O(M(n))$ for positive integer $k$, we may conclude that $f(n) = O(M(\beta) \log \beta)$. Hence the total running time if $\Theta(M(\beta) \log \beta)$.

  \section{[TC] Problem 31.2-4}
  \begin{codebox}
  \Procname{$\proc{Euclid}(a, b)$}
  \li \While $b \neq 0$
  \li \Do $t \gets a \bmod b$
  \li     $a \gets b$
  \li     $b \gets t$
      \End
  \li \Return $a$
  \end{codebox}
  
  \section{[TC] Problem 31.2-5}
  By Lam\'{e}'s theorem, the call $\proc{Euclid}(a, b)$ makes at most $k$ recursive calls, where $b < F_{k+1}$. Since $F_{k+1} > \phi^{k-1}$ when $k > 1$, $b < \phi^{k-1}$ implies $b < F_{k+1}$, so we can take $k = 1 + \log_{\phi} b$ when $b > 1$. If $b = 1$, it only makes one recursive call. Hence $\proc{Euclid}(a, b)$ makes at most $1 + \log_{\phi} b$ recursive calls.

  The executions of $\proc{Euclid}(a, b)$ and $\proc{Euclid}(a/\gcd(a, b), b/\gcd(a, b))$ are same except that the numbers in the former one is $\gcd(a, b)$ times larger than those in the latter one. Hence they make the same number of recursive calls. This improves the bound to $1 + \log_{\phi}(b/\gcd(a,b))$. 
  
  \section{[TC] Problem 31.2-6}
  It returns $(1, (-1)^k F_{k-1}, (-1)^{k+1} F_k)$. We prove this by induction. \par
  For the base case, $k = 1$, the algorithm returns $(1, 1, 0) = (1, (-1)^k F_{k-1}, (-1)^{k+1} F_k)$. \par
  For the induction space, let $k' = k + 1$, the algorithm returns 
  \begin{align*}
    &(1, (-1)^{k+1} F_k, (-1)^k F_{k-1} - \lfloor F_{k+2} / F_{k+1} \rfloor (-1)^{k+1} F_k) \\
  =& (1, (-1)^{k'} F_{k'-1}, (-1)^k (F_{k-1} + F_k)) \\
  =& (1, (-1)^{k'} F_{k'-1}, (-1)^{k'+1} F_{k'})
  \end{align*}
  By mathematical induction, the conclusion is correct.
  
  \section{[TC] Problem 31.2-9}
  $\gcd(n_1n_2, n_3n_4) = 1$ implies $n_i$ and $n_j$ are relatively prime, where $i = 1, 2$, $j = 3, 4$. $\gcd(n_1n_3, n_2n_4) = 1$ implies $n_i$ and $n_j$ are relatively prime, where $i = 1, 3$, $j = 2, 4$. So $n_1, n_2, n_3, n_4$ are pairwise relatively prime.

  Let $p_i (1 \leq i \leq \lceil \lg k \rceil)$ ($q_i$, resp.) denote the product of $n_j$, where the $i$-th bit of the binary representation of $j-1$ is 1 (0, resp.). Then $n_1, n_2, \cdots, n_k$ are pairwise relatively prime if and only if $p_i, q_i (1 \leq i \leq \lceil \lg k \rceil)$ are relatively prime:

  If: for every pair of numbers in $n_1, n_2, \cdots, n_k$, let $n_i, n_j$ denote them. Since $i \neq j$, the binary representation of $i$ and $j$ must differ at at least one bit. Let $a$ denote the index of the bit, then $n_i$ is a term of  $p_a$ (or $q_a$, resp.), $n_j$ is a term of $q_a$ (or $p_a$, resp.). Since $p_a$ and $q_a$ are relatively prime, $n_i$ and $n_j$ are also relatively prime. Hence $n_1, n_2, \cdots, n_k$ are pairwise relatively prime.

  Only if: suppose to the contrary that $p_i$ and $q_i$ are not relatively prime. Let $r$ be any prime factor of $\gcd(p_i, q_i)$. Then, at lest one term in $p_i$ (and $q_i$) is a multiple of $r$. Let $n_x$ be the term in $p_i$, $n_y$ be the term in $q_i$. Since the the binary representation of $x$ and $y$ differ at the $i$-th bit, we have $x \neq y$, but $n_x$ and $n_y$ are multiples of $r$, which means they are not relatively prime, leading to contradiction. So $p_i, q_i (1 \leq i \leq \lceil \lg k \rceil)$ are relatively prime.

  \section{[TC] Problem 31.3-5}
  If $f_a(x) = f_a(y)$, i.e. $ax \equiv ay \pmod n$. Multiplying $a^{-1}$ on both sides gives $x \equiv y \pmod n$, i.e. $x = y$, so $f_a(x)$ is one-to-one. For every $y \in \mathbb{Z}_n^*$, $f(a^{-1}y) = y$, so the function is onto. Therefore, $f_a(x)$ is a bijection from $y \in \mathbb{Z}_n^*$ to $y \in \mathbb{Z}_n^*$, so it is a permutation of $\mathbb{Z}_n^*$.

  \section{[TC] Problem 31.4-2}
  Since $\gcd(a, n) = 1$, the multiplicative inverse of $a$ modulo $n$ exists. Multiplying $a^{-1}$ on both sides of $ax \equiv ay \pmod n$ gives $x \equiv y \pmod n$. \par
  $2 \times 3 \equiv 2 \times 8 \pmod{10}$, while $3 \neq 8 \pmod{10}$, because $\gcd(2, 10) > 1$.
  
  \section{[TC] Problem 31.4-3}
  This will work. Let $x_0' = x'(b/d) \bmod (n/d)$, then $x_0' = x_0 \bmod (n/d) = x_0 + k(n/d)$, which is also a solution. By Theorem 31.24, this will work.

  \section{[TC] Problem 31.5-2}
  The equation system is 
  \begin{align*}
    x &\equiv 1 \pmod 9 \\
    x &\equiv 2 \pmod 8 \\
    x &\equiv 3 \pmod 7
  \end{align*}
  then we have $a_1 = 1, n_1 = 9, m_1 = 56, c_1 = m_1(m_1^{-1} \bmod n_1) = 56 \times 5 = 280$, $a_2 = 2, n_2 = 8, m_2 = 63, c_2 = m_2(m_2^{-1} \bmod n_2) = 63 \times 7 = 441$, $a_3 = 3,  n_3 = 7, m_3 = 72, c_3 = m_3(m_3^{-1} \bmod n_3) = 72 \times 4 = 288$. By the Chinese remainder theorem, we have
  \begin{align*}
    x &\equiv a_1c_1 + a_2c_2 + a_3c_3 & \pmod {504} \\
    x &\equiv 1 \times 280 + 2 \times 441 + 3 \times 288 & \pmod {504} \\
    x &\equiv 2026 & \pmod {504} \\
    x &\equiv 10 & \pmod {504} 
  \end{align*}
  So the solutions are $x = 10 + 504k$, where $k$ is an integer.
  
  \section{[TC] Problem 31.5-3}
  Since $\gcd(a, n) = 1$, we have $\gcd(a, n_i) = 1$, so the multiplicative inverse of $a$ modulo $n_i$ exists. Because $a_i = a \bmod n_i$, we have $a_i^{-1} \equiv a^{-1} \pmod {n_i}$. Hence we have the correspondence
   $$ (a^{-1} \bmod n) \leftrightarrow ((a_1^{-1} \bmod n_1),\cdots, (a_k^{-1} \bmod n_k)). $$
   
  \section{[TC] Problem 31.6-2}
  \begin{codebox}
  \Procname{$\proc{Modular-Exponentiation}(a, b, n)$}
  \li $c = a$
  \li $d = 1$
  \li let $\langle b_k, b_{k-1}, \cdots b_0 \rangle$ be the binary representation of $b$
  \li \For $i \gets 0 \To k$
  \li \Do \If $b_i \isequal 1$
  \li     \Do $d \gets (d \cdot c) \bmod n$
          \End
  \li     $c \gets (c \cdot c) \bmod n$
      \End
  \li \Return $d$
  \end{codebox}
  
  \section{[TC] Problem 31.6-2}
  By Euler's theorem, $a^{\phi(n)-1} \cdot a \equiv 1 \pmod n$, so $a^{\phi(n)-1} \bmod n$ is the multiplicative inverse of $a$ modulo $n$, i.e. $a^{-1} \bmod n = a^{\phi(n)-1} \bmod n$, which can be calculated by \proc{Modular-Exponentiation}.
  
\end{document}
