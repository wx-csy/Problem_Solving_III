\documentclass[a4paper,11pt,twocolumn]{article}
\usepackage{fancyhdr}
\usepackage{enumerate}
\usepackage{times}
\usepackage{mathptmx}
\usepackage{amsmath}
\usepackage{amsfonts}
\usepackage{amssymb}
\usepackage[top=2cm, bottom=2cm, left=2cm, right=2cm]{geometry}

\setlength{\columnsep}{7mm}

\newcommand{\homeworkno}{3.15}
\pagestyle{fancy}
\lhead{Problem Solving: Homework \homeworkno}
\chead{}
\rhead{Chen Shaoyuan (161240004)}
\lfoot{}
\cfoot{\thepage}
\rfoot{}

\allowdisplaybreaks[4]
\renewcommand{\labelenumi}{\textbf{\emph{\alph{enumi}}.}}
\begin{document}
  \title{Problem Solving: Homework \homeworkno}
  \author{Name: Chen Shaoyuan \and Student ID: 161240004}
  \maketitle

  \section{[TJ] Problem 8.6}
  \begin{enumerate}[(a)]
    \item The minimum distance is 2, between (011010) and (011100).
    \item The minimum distance is 1, between (011011) and (111011).
    \item The minimum distance is 1, between (110101) and (110001).
    \item The minimum distance is 2, between (00110110) and (0111100).
  \end{enumerate}
  We hope that the minimum distance for the code is as large as possible, so the error detection and error correction capability is maximal.

  \section{[TJ] Problem 8.7}
  \begin{enumerate}[(a)]
    \item The null space is
    $$ \{(00000), (00101), (10011), (10110)\}. $$
    It is a (5, 2)-block code. A generator matrix is
    $$ \begin{pmatrix} 1 & 0 \\ 0 & 0 \\ 1 & 1 \\ 1 & 0 \\ 0 & 1 \end{pmatrix} .$$
    The generator matrix is not unique.
    \item The null space is
    $$ \{(000000), (010111), (101101), (111010)\}. $$
    It is a (6, 2)-block code. A generator matrix is
    $$ \begin{pmatrix} 1 & 1 \\ 1 & 0 \\ 1 & 1 \\ 0 & 1 \\ 1 & 0 \\ 0 & 1 \end{pmatrix}. $$
    The generator matrix is not unique.
    \item The null space is
    \begin{multline*}
     \{(00000), (00011), (00100), (00111), \\ (11001), (11010), (11101), (11110)\}.
    \end{multline*}
    It is a (5, 3)-block code. A generator matrix is
    $$ \begin{pmatrix}
      1 & 0 & 0 \\
      1 & 0 & 0 \\
      0 & 1 & 0 \\
      0 & 0 & 1 \\
      1 & 0 & 1
    \end{pmatrix}. $$
    The generator matrix is not unique.
    \item The null space is
    \begin{multline*}
      \{(0000000), (0001111), (0010110), (0011001), \\
        (0100101), (0101010), (0110011), (0111100), \\
        (1000011), (1001100), (1010101), (1011010), \\
        (1100110), (1101001), (1110000), (1111111)\}
    \end{multline*}
    It is a (7, 4)-block code. A generator matrix is
    $$ \begin{pmatrix}
      1 & 0 & 0 & 0 \\
      0 & 1 & 0 & 0 \\
      0 & 0 & 1 & 0 \\
      0 & 0 & 0 & 1 \\
      0 & 1 & 1 & 1 \\
      1 & 0 & 1 & 1 \\
      1 & 1 & 0 & 1
    \end{pmatrix}. $$
    The generator matrix is not unique.
  \end{enumerate}

  \section{[TJ] Problem 8.8}
  The code is generated by
  $$ \begin{pmatrix} 1 & 0 \\ 0 & 1 \\ 1 & 0 \\ 0 & 1 \\ 1 & 1 \end{pmatrix}. $$
  Since the minimum distance of this code is 3, this code can be used to either correct 1-bit errors , or detect 1- or 2-bit errors.

  \section{[TJ] Problem 8.9}
  Since the minimum distance of this code is 3 , it can be used to correct 1-bit error. Only the first block of the message can be decoded, and the decoded message is 01101. The other three blocks of message cannot be decoded, for there exist multiple codewords to which the distance is minimum.

  \section{[TJ] Problem 8.11}
  \begin{enumerate}[(a)]
    \item It is a canonical parity-check matrix. The generator matrix is
        $$ \begin{pmatrix} 1 \\ 1 \\ 0 \\ 0 \\ 1 \end{pmatrix}. $$
        Since the minimum distance of this code is 3, this code can be used to either correct 1-bit errors, or detect 1- or 2-bit errors.
    \item It is a canonical parity-check matrix. The generator matrix is
        $$ \begin{pmatrix} 1 & 0 \\ 0 & 1 \\ 0 & 1 \\ 1 & 1 \\ 0 & 1 \\ 1 & 1 \end{pmatrix} $$
        Since the minimum distance of this code is 3, this code can be used to either correct 1-bit errors, or detect 1- or 2-bit errors.
    \item It is a canonical parity-check matrix. The generator matrix is
        $$ \begin{pmatrix} 1 & 0 \\ 0 & 1 \\ 1 & 1 \\ 1 & 0 \end{pmatrix} $$
        Since the minimum distance of this code is 2, this code can be used to detect 1-bit errors.
  \end{enumerate}
  
  \section{[TJ] Problem 8.13}
  \begin{enumerate}[(a)]
    \item $$ \begin{pmatrix} 0 \\ 0 \\ 1 \end{pmatrix} $$.
    \item $$ \begin{pmatrix} 1 \\ 0 \\ 1 \end{pmatrix} $$.
    \item $$ \begin{pmatrix} 1 \\ 1 \\ 1 \end{pmatrix} $$.
    \item $$ \begin{pmatrix} 0 \\ 1 \\ 1 \end{pmatrix} $$.
  \end{enumerate}
  
  \section{[TJ] Problem 8.18}
  Assume that the codewords are determined by the null space of matrix $H$. Let $A = \{\alpha_1, \alpha_2, \cdots, \alpha_m\}$ be a basis of the null space of $H$. Then every codeword is a linear combination of the basis vectors: let $x$ be arbitrary vector in $\mathbb{Z}_2^{m}$, then every codeword $c$ can be written as
  $$ c = Ax $$
  and the $i$th coordinate of the codeword is
  $$ c_i = A_i x$$
  where $A_i$ is the $i$th row of $A$. \par
  If $A_i = 0$, then the $i$th coordinates of all codewords are 0. Otherwise, let $k$ denote the number of 1 in $A_i$. Then, the number of codewords whose $i$th coordinates are 0 is
  $$ 2^{m-k} \sum_{i = 0}^{\lfloor k/2 \rfloor} \binom{k}{2i} = 2^{m-k}2^{k-1} = 2^{m-1} $$
  so half of the codewords have 0 in their $i$th coordinate, and half have 1. Therefore, either the $i$th coordinates in the codewords are all zeros or exactly half of them are.

  \section{[TJ] Problem 8.19}
  Assume that the codewords are determined by the null space of matrix $H$. Let $A = \{\alpha_1, \alpha_2, \cdots, \alpha_m\}$ be a basis of the null space of $H$, and thus every codeword $c$ can be written as $c = Ax$. Let $k$ denote the basis vectors whose weights are odd. If no such basis vector exists, then all codewords have even weight. Otherwise, the number of codewords whose weight is even is
  $$ 2^{m-k} \sum_{i = 0}^{\lfloor k/2 \rfloor} \binom{k}{2i} = 2^{m-k}2^{k-1} = 2^{m-1} $$
  so half of the codewords have even weights, and half have odd. Therefore, either every codeword has even weight or exactly half of the codewords have even weight.

  \section{[TJ] Problem 8.21}
  A $11 \times 7$ generator matrix must be used to transmit the 128 ASCII characters, while a $12 \times 8$ matrix for extended ASCII character set of 256 characters. In both cases, 4 extra bits are necessary to locate the error, if exists. If only error detection are requires, the sizes of matrices are $8 \times 7$ and $9 \times 8$, respectively, by simply adding a parity bit.

  \section{[TJ] Problem 8.22}
  The canonical parity-check matrix that gives the even parity check bit code with three information positions is
  $$ \begin{pmatrix} 1 & 1 & 1 & 1 \end{pmatrix}, $$
  while the matrix for seven information positions is
  $$ \begin{pmatrix} 1 & 1 & 1 & 1 & 1 & 1 & 1 & 1 \end{pmatrix}. $$
  The corresponding standard generator matrices are 
  $$ \begin{pmatrix} I_3 \\ 1_{1 \times 3}  \end{pmatrix} $$
  and
  $$ \begin{pmatrix} I_7 \\ 1_{1 \times 7}  \end{pmatrix}, $$
  respectively.
  
  \section{[TJ] Problem 8.23}
  5. 6.
\end{document}
