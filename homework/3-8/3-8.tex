\documentclass[a4paper,11pt,twocolumn]{article}
\usepackage{fancyhdr}
\usepackage{enumerate}
\usepackage{times}
\usepackage{mathptmx}
\usepackage{amsmath}
\usepackage{amsfonts}
\usepackage{amssymb}
\usepackage{clrscode3e}
\usepackage[top=2cm, bottom=2cm, left=2cm, right=2cm]{geometry}

\setlength{\columnsep}{7mm}

\newcommand{\homeworkno}{3.7}
\pagestyle{fancy}
\lhead{Problem Solving: Homework \homeworkno}
\chead{}
\rhead{Chen Shaoyuan (161240004)}
\lfoot{}
\cfoot{\thepage}
\rfoot{}

\allowdisplaybreaks[4]
\renewcommand{\labelenumi}{\textbf{\emph{\alph{enumi}}.}}
\begin{document}
  \title{Problem Solving: Homework \homeworkno}
  \author{Name: Chen Shaoyuan \and Student ID: 161240004}
  \maketitle

  \section{[GC] Problem 29.1-4}
  \begin{align*}
    \text{maximize} && -2x_1'  +  2x_1''  -  7x_2  +  x_3' \\
    \text{subject to} \\
    && -x_1'  +  x_1''  -  x_3'  &\leq  7 \\
    && x_1'  -  x_1''  +  x_3'  &\leq  7 \\
    && -3x_1'  +  3x_1''  -  x_2  &\leq  24 \\
    && x_1',  x_1'',  x_2,  x_3'  &\geq  0\\
    \text{where}  \\
    && x_1' - x_1'' &= x_1\\
    && x_3' &= -x_3
  \end{align*}

  \section{[GC] Problem 29.1-5}
  \begin{align*}
    z & = 2x_1 - 6x_3 \\
    x_4 &= 7 - x_1 - x_2 + x_3 \\
    x_5 &= -8 + 3x_1 - x_2\\
    x_6 &= -x_1 + 2x_2 + 2x_3
  \end{align*}

  \section{[GC] Problem 29.1-6}
  The second constraint implies $x_1 + x_2 \geq 5$, which contradicts the first one. So this linear program is infeasible.

  \section{[GC] Problem 29.1-7}
  Let $x_1 = 2t, x_2 = t$. It is easy to verify, that the solution $(x_1, x_2)$ is feasible if $t>1$. Therefore, the objective function $z = x_1 - x_2 = 2t - t  = t$ can be arbitrarily large, and this linear program is unbounded.

  \section{[GC] Problem 29.1-9}
  \begin{align*}
    \text{minimize} && x_1 + x_2 \\
    \text{subject to} \\
    && x_1, x_2 & \geq 0
  \end{align*}
  In this program, the feasible region is obviously unbounded, but it has finite optimal objective value $0$.

  \section{[GC] Problem 29.2-2}
  \begin{align*}
    \text{minimize} && d_y \\
    \text{subject to} \\
    && d_s & = 0 \\
    && d_t - d_s & \leq 3 \\
    && d_y - d_s & \leq 5 \\
    && d_t - d_y & \leq 1 \\
    && d_y - d_t & \leq 2 \\
    && d_x - d_t & \leq 6 \\
    && d_x - d_y & \leq 4 \\
    && d_z - d_y & \leq 6 \\
    && d_s - d_z & \leq 3 \\
    && d_x - d_z & \leq 7 \\
    && d_z - d_x & \leq 2
  \end{align*}

  \section{[GC] Problem 29.2-3}
  \begin{align*}
    \text{minimize} && \sum_{v \in V} d_v \\
    \text{subject to} \\
    && d_s & = 0 \\
    && d_v - d_u & \leq w(u, v), \forall (u, v) \in E
  \end{align*}

  \section{[GC] Problem 29.2-6}
  \begin{align*}
    \text{maximize} && \sum_{(u, v) \in E} f_{uv} \\
    \text{subject to} \\
    && f_{uv} & \geq 0, \forall (u, v) \in E \\
    && \sum_{u \in V} f_{uv} & \leq 1, \forall v \in V \\
    && \sum_{v \in V} f_{uv} & \leq 1, \forall u \in V
  \end{align*}

  \section{[GC] Problem 29.3-2}
  Line 14 of \proc{Pivot} gives the new value of $v$: $\hat{v} = v + c_e \hat{b}_e$. Line 14 of \proc{Simplex} guarantees that $c_e$ is positive, while $\hat{b}_e$ is defined as $b_l / a_{le}$. Note that the index $l$ chosen in line 9 of \proc{Simplex} must satisfy that $\Delta_{i} \neq \infty$, which means $a_{le}$ is positive. $b_l$ is nonnegative, for otherwise the program is infeasible. Therefore, the call to \proc{Pivot} in \proc{Simplex} never decreases the value of $v$.

  \section{[GC] Problem 29.3-2}
  First, the procedure \proc{Pivot} do not changes the nonnegativity constraints, which are still $x_i > 0, \forall i \in N \cup B$. Second, it solves one equality for the entering variable, and substitutes the entering variable in the target function, which does not change the function. Then, it replaces the equality solved before with its solution form, and substitutes the solution for the entering variable in all other equalities and target functions. In the new equalities, if we solve the equality containing the original entering variable for the original leaving variable, and do substitutions like before, we will get the original equalities and target function. Therefore, the slack form given to \proc{Pivot} and the one it returns are equivalent.

  \section{[GC] Problem 29.3-5}
  First, convert it to slack form:
  \begin{align*}
    z &= 18x_1 + 12.5x_2 \\
    x_3 &= 20 - x_1 - x_2 \\
    x_4 &= 12 - x_1 \\
    x_5 &= 16 - x_2
  \end{align*}
  Choose $x_1$ as entering variable and $x_4$ as leaving variable, and perform a pivot:
  \begin{align*}
    z &= 216 - 18x_4 + 12.5x_2 \\
    x_3 &= 8 - x_2 - x_4 \\
    x_1 &= 12 - x_4 \\
    x_5 &= 16 - x_2
  \end{align*}
  Choose $x_2$ as entering variable and $x_3$ as leaving variable, and perform a pivot:
  \begin{align*}
    z &= 316 - 12.5x_3 - 30.5x_4 \\
    x_2 &= 8 - x_3 - x_4 \\
    x_1 &= 12 - x_4 \\
    x_5 &= 24 - x_3 - x_4
  \end{align*}
  Now, all coefficients of the target function is not positive, so the algorithm terminates. The optimal solution is $(x_1, x_2) = (12, 8)$, and the optimal value is 316.

  \section{[GC] Problem 29.4-2}
  Consider a linear program which only contains greater-than-or-equal-to constraints:
  \begin{align*}
    \text{max./min.} && \sum_{j = 1}^{n} c_j x_j \\
    \text{subject to} \\
    && \sum_{j = 1}^{n} a_{ij} x_j \geq b_i, \text{ for } 0 < i \leq m, \\
  \end{align*}
  We first convert it to standard form:
  \begin{align*}
    \text{max./min.} && \sum_{j = 1}^{n} c_j (x_j - x_j') \\
    \text{subject to} \\
    && \sum_{j = 1}^{n} a_{ij} (x_j - x_j') \geq b_i, \text{ for } 0 < i \leq m, \\
    && x_j, x_j' \geq 0
  \end{align*}
  Then, take the dual of the program:
  \begin{align*}
    \text{min./max.} && \sum_{i = 1}^{m} b_iy_i \\
    \text{subject to} \\
    && \sum_{i = 1}^{m} a_{ij} y_i \leq 0, \text{ for } 0 < j \leq n \\
    && \sum_{i = 1}^{m} -a_{ij} y_i \leq 0, \text{ for } 0 < j \leq n \\
    && y_i \geq 0, \text{ for } 0 < i \leq m
  \end{align*}
  which can be rewritten as
  \begin{align*}
    \text{min./max.} && \sum_{i = 1}^{n} b_iy_i \\
    \text{subject to} \\
    && \sum_{i = 1}^{m} a_{ij} y_i = 0, \text{ for } 0 < j \leq n \\
    && y_i \geq 0, \text{ for } 0 < i \leq m
  \end{align*}
  Likewise, the dual of a linear program which only contains greater-than-or-equal-to constraints:
  \begin{align*}
    \text{max./min.} && \sum_{j = 1}^{n} c_j x_j \\
    \text{subject to} \\
    && \sum_{j = 1}^{n} a_{ij} x_j \geq b_i, \text{for } 0 < i \leq m, \\
  \end{align*}
  is
  \begin{align*}
    \text{min./max.} && \sum_{i = 1}^{n} b_iy_i \\
    \text{subject to} \\
    && \sum_{i = 1}^{m} a_{ij} y_i = 0 , \text{ for } 0 < j \leq n \\
    && y_i \leq 0, \text{ for } 0 < i \leq m
  \end{align*}
  If a linear program only contains equality constraints,
  \begin{align*}
    \text{max./min.} && \sum_{j = 1}^{n} c_j x_j \\
    \text{subject to} \\
    && \sum_{j = 1}^{n} a_{ij} x_j = b_i, \text{for } 0 < i \leq m, \\
  \end{align*}
  its dual program is
  \begin{align*}
    \text{min./max.} && \sum_{i = 1}^{n} b_i(y_i + y_i') \\
    \text{subject to} \\
    && \sum_{i = 1}^{m} a_{ij} (y_i + y_i') = 0 , \text{ for } 0 < j \leq n \\
    && y_i \geq 0, y_i' \leq 0, \text{ for } 0 < i \leq m
  \end{align*}
  or, equivalently,
  \begin{align*}
    \text{min./max.} && \sum_{i = 1}^{n} b_i y_i \\
    \text{subject to} \\
    && \sum_{i = 1}^{m} a_{ij} y_i = 0 , \text{ for } 0 < j \leq n
  \end{align*}

  Combine these three cases, we obtain the method of taking the dual of arbitrary linear program directly. For arbitrary linear program
  \begin{align*}
    \text{max./min.} && \sum_{j = 1}^{n} c_j x_j \\
    \text{subject to} \\
    && \sum_{j = 1}^{n} a_{ij} x_j \leq b_i, \text{ for } 0 < i \leq m_1, \\
    && \sum_{j = 1}^{n} a_{ij} x_j \geq b_i, \text{ for } m_1 < i \leq m_2, \\
    && \sum_{j = 1}^{n} a_{ij} x_j = b_i, \text{ for } m_2 < i \leq m_3
  \end{align*}
  its dual program is
  \begin{align*}
    \text{min./max.} && \sum_{i = 1}^{m_3} b_iy_i \\
    \text{subject to} \\
    && \sum_{i = 1}^{m_3} a_{ij} y_i  = 0, \text{ for } 0 < j \leq n \\
    && y_i \geq 0, \text{ for } 0 < i \leq m_1 \\
    && y_i \leq 0, \text{ for } m_1 < i \leq m_2 \\
  \end{align*}

  \section{[GC] Problem 29-1}
  \begin{enumerate}
    \item We can arbitrarily define a target function and run the algorithm for linear programming. If it returns an optimal solution or reports the program is unbounded, then the inequalities are feasible; otherwise, they are infeasible.
    \item First, we use the algorithm linear-inequality feasibility algorithm to get feasible solutions for the linear program and its dual. Second, calculate the values of target functions $v_l, v_r$, and the optimal value must between the two values. Third, we apply binary search, that is, add a constraint, $z < (v_l+v_r)/2$, to guess the optimal value. If the inequalities are satisfiable, then replace $v_l$ with $(v_l+v_r)/2$ if the target function is to be maximized, or $v_r$ if is to be minimized; otherwise, replace $v_r$ or $v_l$ with $(v_l+v_r)/2$. Repeatedly doing this until the error is admissible, the final feasible solution is optimal.
  \end{enumerate}
\end{document}
